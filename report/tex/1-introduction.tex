\begin{center}
    \textbf{ВВЕДЕНИЕ}
\end{center}
\addcontentsline{toc}{chapter}{ВВЕДЕНИЕ}

В современном мире компьютерная графика является важным инструментом для создания реалистичных изображений в различных областях --- от компьютерных игр и кино до научных исследований и инженерных задач. Одной из ключевых задач графики является моделирование и визуализация трехмерных объектов с высокой степенью детализации. Это особенно актуально при создании сложных объектов, требующих учета текстур, неровностей поверхности и взаимодействия с внешними условиями.

Одним из эффективных подходов для создания реалистичных моделей является визуализация тел вращения --- объектов, форма которых определяется вращением плоской кривой (образующей) вокруг оси (направляющей). Для более детальной и правдоподобной визуализации на полученное тело можно нанести фактуру и текстуру материала. Это требует применения специализированных алгоритмов, таких как рельефное текстурирование, что позволяет сделать модель более естественной и реалистичной.

Цель работы --- разработка программного обеспечения для визуализации тел вращения с добавлением фактуры и текстуры материала. Для достижения поставленной цели необходимо решить следующие задачи:
\begin{itemize}
    \item[---] описать предметную область работы;
    \item[---] рассмотреть и выбрать алгоритмы построения реалистичного тела вращения;
    \item[---] на основе выбранных алгоритмов спроектировать программное обеспечение;
    \item[---] реализовать спроектированное программное обеспечение;
    \item[---] провести исследование на основе разработанной программы.
\end{itemize}